% !TeX root = 高等微积分(2).tex

\section{一阶ODE初值问题解的存在性和唯一性}

\begin{example}
  未必有长期解.
  \begin{equation}
    \begin{cases}
      y' = y^{2}, \\
      y(0) = y_0 > 0
    \end{cases}
  \end{equation}
\end{example}

\begin{example}
  解未必唯一.
  \begin{equation}
    \begin{cases}
      y' = \sqrt{\left| y \right|}, \\
      y(0) = 0.
    \end{cases}
  \end{equation}
  在$x=0$附近有无穷个解.

  构造解:让$x>0$时$y>0$.
  \begin{equation}
    \int \frac{\mathrm{d} y}{\sqrt{y}} = \int_{0}^{x} \mathrm{d}x = x- b.
  \end{equation}
  即
  \begin{equation}
   \left. 2 y^{\frac{1}{2}}\right|_{0}^{y(x)} = x-b.
  \end{equation}
  于是
  \begin{equation}
    y = 
    \begin{cases}
      \frac{\left( x-b \right)^{2}}{4}, & x>b \\
      \ 0 , & -a \le x \le b, \\
      - \frac{\left( x+a \right)^{2}}{4}, & x< -a.
    \end{cases}
  \end{equation}
\end{example}

对怎样好的ODE初值问题, 其短期解是存在且唯一的?

\begin{proposition}
  $C^{1}$光滑函数$y(x)$满足
  \begin{equation}
    \begin{cases}
      y' = f(x,y), \\
      y(x_0) = y_0.
    \end{cases}
  \end{equation}
  等价于$\forall x$有 $y(x) = y_0 + \int _{x_0}^{x} f(t,y(t))\, \mathrm{d} t$.
\end{proposition}

\begin{proof}
  设$y \in C^{1}$是它的解, 
  \begin{equation}
    \int _{x_0}^{x} y'\, \mathrm{d} x = \int _{x_0}^{x} f(x,y) \, \mathrm{d} x
  \end{equation}
  即
  \begin{equation}
    y(x) = y_0 + \int_{x_0}^{x} f(x,y(x)) \, \mathrm{d}x .
  \end{equation}


  另一方面, 设$y(x)$满足积分方程, 则$y(x_0) = y_0$.
  利用连续函数变上限积分的求导法则可以得到,
  \begin{equation}
    \frac{\mathrm{d}}{\mathrm{d} x} y(x) = \frac{\mathrm{d} }{\mathrm{d} x} \left( y_0 + \int_{x_0}^{x} f(t,y(t)) \, \mathrm{d}t  \right) = f(x,y(x)).
  \end{equation}
  表明$y(x)$可导, 且$y' = f(x,y)$也连续, $y \in C^{1}$.
\end{proof}
上面的命题将微分方程转化为积分方程.
\begin{example}[含时微扰论]
  \begin{equation}
    \begin{cases}
      \mathrm{i} \frac{\mathrm{d} U}{\mathrm{d} t} = V(t) U(t), \\
      U(-\infty ) = \operatorname{Id}
    \end{cases}
  \end{equation}
  这里面的$U,V$都是无穷维Hilbert空间中的算子. 改写为积分方程得到
  \begin{equation}
    U(t) = 1 + \int_{-\infty }^{t} -\mathrm{i} U V(t) \, \mathrm{d}t 
  \end{equation}
  迭代即可得到Dyson级数,
  \begin{equation}
    \begin{aligned}
      U(t) & = 1 + \int_{-\infty}^{t} -\mathrm{i} V(t_1) \, \mathrm{d}t_1 + \int _{-\infty }^{t}\int _{-\infty }^{t_1} \left( -\mathrm{i}V(t_1) \right) \left( -\mathrm{i}V(t_2) \right)\, \mathrm{d} t_2 \mathrm{d} \,t_1 +\cdots  \\
           & \equiv  T\left\{ \exp \left( \int _{-\infty }^{t} V(t) \, \mathrm{d} t \right) \right\}
    \end{aligned}
  \end{equation}
\end{example}

猜想这种逐阶近似$\phi _{n}(x) = y_0 + \int _{x_0}^{x} f(t,\phi _{n-1}(x)) \, \mathrm{d} t$得到的解$\left\{ \phi _{n}(x) \right\}$有极限.采用压缩映像定理来证明.
\begin{theorem}
  设$\left( X,d \right)$是一个度量空间\footnote{满足正定, 对称, 三角不等式.}, 且任何Cauchy序列都收敛.

  设$T\colon X \to X$是压缩的, 即$\exists 0 < c < 1$使得$\forall x,y \in X$有$d\left( T(x), T(y) \right) \le c d(x,y)$.

  则$T$有唯一的不动点.
\end{theorem}
压缩映像定理有更强的版本.
\begin{theorem}[Weissinger]
  设$(X,d)$是完备度量空间, 设$T\colon X \to X$满足$\forall n \in \mathbb{Z}_{+}$存在常数$\theta_{n}$使$\forall x,y$有
  \begin{equation}
    d\left( T^{(n)}(x), T^{(n)}(y) \right) \le \theta_{n} d(x,y)
  \end{equation}
  且
  \begin{equation}
    \sum_{n=1}^{\infty } \theta_{n} < + \infty.
  \end{equation}
  则$T$有唯一的不动点.
\end{theorem}
\begin{proof}[Proof Weissinger Theorem]
  选取$x_0 \in X$, 定义点列
  \begin{equation}
  \left\{ x_n = T^{(n)}(x_0) \right\}_{n=1}^{+\infty }
  \end{equation}
  来证$\{ x_n \}$是Cauchy列, 即$\forall \varepsilon > 0$, $\exists N \in \mathbb{Z}_+$, $\forall m > n \ge N$有 $d(x_m, x_n) < \varepsilon$
  \begin{equation}
    d\left( x_m, x_n \right) = d \left( T^{n} (T^{m-n}x_0), T^{n} (x_0) \right) \le \theta_{n} d \left( T^{m-n}(x_0), x_0 \right)   \end{equation}
  由三角不等式, $x_{m-n}$与$x_0$的距离小于等于$\sum_{i=0}^{m-n-1}d(x_{i}, x_{i+1}) $.于是
  \begin{equation}
    d\left( x_m,x_n \right) \le \theta_{n} \left( 1 + \theta_1 + \cdots  + \theta_{m-n-1} \right) d\left( x_0, x_1 \right)
  \end{equation}
  由于$\{ \theta_{n} \}$的求和是有限的, $\lim _{n \to + \infty } \theta_{n} = 0$, 得证.
\end{proof}
Weissinger定理的适用范围更广, 因为$\theta_{1}$可以大于$1$.

\begin{definition}[Lipshitz condition]
  称$f$在$I$上满足(关于$y_0$)的Lipshitz条件, 如果$\exists \text{正数} L$, 使
  \begin{equation}
    \left| f(x,y_1) - f(x,y_2) \right| \le L \left| y_1 - y_2 \right|, \quad \forall (x,y_1), (x,y_2) \in I.
  \end{equation}
\end{definition}

\begin{theorem}[Picard-Lindel\"of定理]
  求解初值问题
  \begin{equation}
    \begin{cases}
      y' = f(x,y), \\
      y(x_0) = y_0.
    \end{cases}
  \end{equation}
  设$f$在$I = [x_0 \pm a] \times [y_0 \pm b]$中连续, 且$f$在$I$上满足Lipshitz条件.

  记$M = \max _{I} |f|$, 则上述ODE初值问题在区间$[x_0-h, x_0+h]$上有解且解唯一, 其中$h = \min \{ a, \frac{b}{M} \}$.
\end{theorem}

\begin{proof}
  Step 1: 找积分方程的不动点, 找$y(x)$, 使得$y(x) = y_0 + \int_{x_0}^{x} f(t,y(t)) \, \mathrm{d}t $, 视右边为$T(y)$.

  定义$X$为所有连续函数$\phi\colon [x_0-h,x_0+h] \to \mathbb{R}$构成的集合, 满足$\phi (x_0) = y_0$, 且$\phi (x) \in [y_0-b, y_0+b], \ \forall x \in  [x_0-h,x_0+h]$

  定义$T\colon X \to X$为
  \begin{equation}
    \left( T(\phi ) \right)(x) = y_0 + \int_{x_0}^{x} f(t,\phi (t)) \, \mathrm{d}t. 
  \end{equation}
  显然$T(\phi )$在$X$中.

  Step 2: $X$上有完备度量.
  定义$X$上度量
  \begin{equation}
    d(\phi ,\psi ) \equiv \max_{x \in [x_0-h,x_0+h]} \left| \phi (x) - \psi (x) \right|
  \end{equation}
  逐点距离的最大值$l_{\infty } \operatorname{norm}$.

  可验证$(X,d)$是完备的度量空间.

  Step 3: 验证$T$满足Weissinger定理的条件.
  \begin{equation}
    d\left( T^{n}\phi , T^{n}\psi  \right) = \max \left| T^{n}\phi(x) - T^{n}\psi(x)  \right|.
  \end{equation}
  所以
  \begin{equation}
    \begin{aligned}
  & \left| T^{n}\phi(x) - T^{n}\psi (x) \right| = \left| T T^{n-1}\phi (x) - TT^{n-1} \phi (x) \right|\\
      = & \left| \int_{x_0}^{x} \left( f\left( t, T^{n-1}\phi (t) \right) - f\left( t, T^{n-1}\psi (t) \right) \right) \, \mathrm{d}t  \right| \\
    \le & \left|  \int_{x_0}^{x} \left| f\left( t, T^{n-1}\phi (t) \right) - f\left( t, T^{n-1}\psi (t) \right) \right| \, \mathrm{d}t \right| \\
    \le & \left| \int_{x_0}^{x} L\left| T^{n-1}\phi (t) - T^{n-1}\psi (t)\right| \, \mathrm{d}t  \right|.
    \end{aligned}
  \end{equation}
  最后一步用到了Lipshitz条件.

  用归纳法可以证明
  \begin{equation}
    \left| T^{n} \phi (x) - T^{n}\psi (x) \right| \le  \frac{L^{n}}{n!} \left| x-x_0 \right|^{n} d(\phi ,\psi )
  \end{equation}

  于是得到
  \begin{equation}
    d\left( T^{n}\phi , T^{n}\psi  \right) \le \frac{L^{n}}{n!} h^{n} d(\phi ,\psi )
  \end{equation}
  取$\theta_{n} = \frac{L^{n}}{n!} h^{n}$, 于是
  \begin{equation}
    \sum_{n=0}^{+\infty } \theta_{n} = \sum \frac{1}{n!} \left( Lh \right)^{n} = \mathrm{e}^{Lh} < +\infty .
  \end{equation}
\end{proof}
