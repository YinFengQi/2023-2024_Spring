% !TeX root = 高等微积分(2).tex
在物理中, 这种场被称为保守场, 做功与路径无关.

\begin{figure}[ht]
    \centering
    \incfig{保守场做功}
    \caption{保守场做功}
    \label{fig:保守场做功}
\end{figure}
对于保守场而言, $\forall L_1, L_2$有
\begin{equation}
  \int _{L_1} \vec{F} \cdot \mathrm{d} \vec{r} = \int _{L_2} \vec{F} \cdot \mathrm{d} \vec{r}.
\end{equation}

\begin{proposition}
  设 $X \subseteq \mathbb{R}^{2}$且$X$无洞, $P, Q$是$X$上的光滑函数, 则 $\exists X \text{上的光滑函数} \phi $, 使得 $\frac{\partial \phi }{\partial x}= P, \frac{\partial \phi }{\partial y} = Q$的充要条件是
  \begin{equation}
    \frac{\partial P}{\partial y} = \frac{\partial Q}{\partial x}.
  \end{equation}
\end{proposition}
\begin{proof}
  分两个方向来证明.

  必要性: 由Schwartz定理, $\phi $ 的光滑性保证了 $\frac{\partial P}{\partial y} = \frac{\partial Q}{\partial x}$.

  充分性: 沿着一条路径构造$\phi (\vec{r}) = \int ^{\vec{r}} \vec{F}\cdot \mathrm{d} \vec{r}$, 用格林公式可以验证与路径无关.
\end{proof}


\subsection{积分因子法}
某些情况下未必有 $\frac{\partial P}{\partial y} = \frac{\partial Q}{\partial x}$, 能否找到一个适当的因子 $\mu \left( x,y \right)$, 使得 $\mu P \mathrm{d} x + \mu Q \mathrm{d} y$ 是全微分? 
称这样的$\mu \left( x,y \right)$为该ODE的积分因子.

积分因子需要满足
\begin{equation}
  \frac{\partial \mu P}{\partial y} = \frac{\partial \mu Q}{\partial x},
\end{equation}
展开得到
\begin{equation}
  \frac{\partial \mu}{\partial y} P + \mu \frac{\partial P}{\partial y} = \frac{\partial \mu}{\partial y} Q + \mu \frac{\partial Q}{\partial x}
\end{equation}
这是一个PDE问题, 我们试图找到一个一元的积分因子$\mu$.

若希望$\mu = \mu(x)$, 则
\begin{equation}
  \mu \frac{\partial P}{\partial y} =\frac{\mathrm{d} \mu}{\mathrm{d} x} Q + \mu \frac{\partial Q}{\partial x}
\end{equation}
即
\begin{equation}
  \frac{\mathrm{d} \mu}{\mathrm{d} x} = \frac{\left( \frac{\partial P}{\partial y} - \frac{\partial Q}{\partial x} \right)}{Q} \mu .
\end{equation}
若$\frac{\frac{\partial P}{\partial y} - \frac{\partial Q}{\partial x}}{Q} = R(x)$, 只依赖于$x$, 则 $\exists \mu = \mu(x)$ 是积分因子, 因为只需 $\frac{1}{\mu} \frac{\mathrm{d} \mu}{\mathrm{d} x} = R(x)$, 是可分离变量的ODE, 解得
\begin{equation}
  \mu (x) = \mathrm{e}^{\int R(x)\, \mathrm{d} x}.
\end{equation}

于是, 若$\frac{\frac{\partial P}{\partial y} - \frac{\partial Q}{\partial x}}{Q} = R(x)$只依赖于$x$, 则 $\mu = \exp \left( \int_{x_0}^{x} R(x) \, \mathrm{d}x  \right)$是ODE的积分因子.

类似地, 若$\frac{\frac{\partial P}{\partial y} - \frac{\partial Q}{\partial x}}{- P} = T(y)$, 则 $\lambda(y) = \exp \left( \int_{y_0}^{y} T(y) \, \mathrm{d}y  \right)$是ODE的积分因子.

\begin{example}
  求解
  \begin{equation}
    - \left( 1-x^{2} + y \right)\mathrm{d} x + x \mathrm{d} y = 0.
  \end{equation}
  不难算出 $\frac{\partial P}{\partial y} - \frac{\partial Q}{\partial x} = -2$, 所以原方程不是全微分, 我们寻找积分因子. 由于
  \begin{equation}
    \frac{\frac{\partial P}{\partial y} - \frac{\partial Q}{\partial x}}{Q} = \frac{-2}{x} = R(x).
  \end{equation}
  所以ODE存在一个积分因子$\mu(x) = \exp \left( \int R(x) \, \mathrm{d} x \right) = \exp \left( \int -\frac{2}{x} \, \mathrm{d} x \right) = \frac{1}{x^{2}}$.

\begin{figure}[ht]
    \centering
    \incfig{ode积分路径}
    \caption{ODE积分路径}
    \label{fig:ode积分路径}
\end{figure}
按照图 \ref{fig:ode积分路径}中的积分路径进行积分, 得到
\begin{equation}
  \phi (x,y) = \int_{1}^{x} \frac{P(x,0)}{x^{2}} \, \mathrm{d}x + \int_{0}^{y} \frac{Q(x,y)}{x^{2}} \, \mathrm{d}y = \int_{1}^{x} \frac{x^{2}-1}{x^{2}} \, \mathrm{d}x + \int_{0}^{y} \frac{1}{x} \, \mathrm{d}y = x-1 + \left( \frac{1}{x} - 1 \right) + \frac{1}{x} y .
\end{equation}
于是等势面为$\phi = \operatorname{const}$
\begin{equation}
  y = \left( C - x - \frac{1}{x} \right)x = - x^{2} - 1 + Cx .
\end{equation}

\end{example}

\subsection{高阶ODE的降阶法}
不显含$y$的二阶ODE, 即
\begin{equation}
  y'' = f(x,y')
\end{equation}
令$y'(x) = z(x)$, 代回得到
\begin{equation}
  z' = g(x,z).
\end{equation}
降为一阶ODE, 解出$z = \phi (x,C)$, 于是$y(x) = \int \phi \, \mathrm{d} x$.

不显含$x$的二阶ODE, 形如$y'' = f(y,y')$的初值问题. 我们不希望用到$x$的区间, 考虑$\phi' \phi ^{-1}$与$\phi'' \phi ^{-1}$.
\begin{proof}[解法]
  如果$\phi'(x_0)\neq 0$, 那么由反函数定理可得, $\phi $在$x_0$附近的小区间内存在逆$\phi ^{-1}$, 考虑$P(y)= \phi' \left( \phi ^{-1} (y) \right)$. 
  用它表示$\phi ''$.
  \begin{equation}
    \frac{\mathrm{d} P}{\mathrm{d} y} = \phi '' \left( \phi ^{-1}(y) \right) \left( \phi ^{-1}(y) \right)' = \phi '' \left( \phi ^{-1}(y) \right) \frac{1}{\phi' \left( \phi ^{-1}(y) \right)}.
  \end{equation}
  即
  \begin{equation}
    \phi '' \left( \phi ^{-1}(y) \right) = P(y) \frac{\mathrm{d} P}{\mathrm{d} y}.
  \end{equation}

  我们回到ODE, 于是$\forall x$有$\phi '' \left( \phi ^{-1}(y) \right) = f\left( \phi \left( \phi ^{-1}(y) \right), \phi ' \left( \phi ^{-1}(y) \right) \right)$, 即
  \begin{equation}
    P(y) \frac{\mathrm{d} P}{\mathrm{d} y} = f(y, P(y))
  \end{equation}
  这是关于$P$的一阶ODE.
\end{proof}
我们形式化地改写上面的解, 令$P(y) = y' \circ y^{-1}$, 则$P \dfrac{\mathrm{d} P}{\mathrm{d} y} = y'' \circ y^{-1}$, 从而原方程变为
\begin{equation}
  P \frac{\mathrm{d} P}{\mathrm{d} y} = f(y, P(y)).
\end{equation}
解出$P(y) = \psi \left( y \right)$, 得到
\begin{equation}
  \phi '(x) = \psi \left( \phi (x) \right),
\end{equation}
于是解出原方程.

进一步改写,
\begin{equation}
  \begin{cases}
    y'' = f(y,y'),\quad \text{(不显含$x$)} \\
    y(x_0) = y_0, \\
    y' (x_0) = y_1 \neq 0.
  \end{cases}
\end{equation}
令$P = y'$, 于是$P \frac{\mathrm{d} P}{\mathrm{d} y} = y''$, 从而
\begin{equation}
  \begin{cases}
    P \frac{\mathrm{d} P}{\mathrm{d} y} = f(y,p), \\
    P(y_0) = y_1.
  \end{cases}
\end{equation}
解出$P(y) = \psi (y)$, 然后求出$y$.

\begin{example}
  求解
  \begin{equation}
    \begin{cases}
      1 + \left( y' \right)^{2} = 2y y'', \\
      y(1) = 1, \\
      y'(1) = -1.
    \end{cases}
  \end{equation}
  令$P = y'$, 则$y'' = P \frac{\mathrm{d} P}{\mathrm{d} y}$, 从而
  \begin{equation}
    \begin{cases}
      1 + P ^{2} = 2y P \frac{\mathrm{d} P}{\mathrm{d} y}, \\
      P(1) = -1.
    \end{cases}
  \end{equation}
  于是
  \begin{equation}
    P(y) = - \sqrt{2y-1}.
  \end{equation}
  两边积分得到
  \begin{equation}
    \int_{1}^{y(x)} \frac{\mathrm{d} y}{\sqrt{2y-1}} = \int _{1}^{x}  -\mathrm{d} x = 1- x.
  \end{equation}
  解得
  \begin{equation}
    y = \frac{x^{2} - 4x + 5}{2}.
  \end{equation}
\end{example}
