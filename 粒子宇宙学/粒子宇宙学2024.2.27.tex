% !TeX root = 粒子宇宙学.tex
\chapter{Why need inflation}
\subsection{Basic Relations}
Natural units: $c=\hbar =k =1$

Some relations:
\begin{equation}
  \begin{gathered}
    \left[ L \right] = \left[ T \right] = -1, \\
    \left[ M \right] = \left[ \text{Temperature} \right] = 1, \\
    1 \mathrm{\mu m} \sim 1 \mathrm{eV}^{-1}, \\
    \text{proton mass} \simeq \mathrm{GeV}, \\
    1\mathrm{s} \sim 10^{8} \mathrm{m} \sim \left( 10^{14} \mathrm{eV}\right)^{-1}, \\
    1 \mathrm{g} \sim 10^{23} \mathrm{GeV} ,\\
    1 \mathrm{K} \sim 10^{-4} \mathrm{eV} .
  \end{gathered}
\end{equation}
Point mass cannot exist in QM, for principle of uncertainty, and in GR for black hole.
QM gives Compton wavelength $\lambda = \frac{1}{m} = \frac{\hbar }{mc} $. GR gives Schwarzschild radius $R \sim \frac{2Gm}{c^{2}}$.
\begin{equation}
  G \sim \frac{1}{\left( 10^{19} \mathrm{GeV} \right)^{2}}
\end{equation}
When $\lambda = R$, we have
\begin{equation}
  m \sim G^{-\frac{1}{2}} \sim 10^{19} \mathrm{GeV}.
\end{equation}

\section{Standard Model}

\begin{figure}[ht]
    \centering
    \incfig{history-of-sm}
    \caption{history of SM}
    \label{fig:history-of-sm}
\end{figure}

\section{Beyond SM}
\begin{itemize}
  \item Neutrino Mass
  \item Matter-Anti-matter asymmetry $(10^{9})$
  \item Dark Matter
  \item Cosmic Inflation ($\sim \mathrm{e}^{Ht}, \ H \sim 10^{14}\mathrm{GeV}$)
\end{itemize}

This course would be focused on cosmic inflation, in which exponentially inflation gives out the maximally symmetric spacetime of constant curvature, i.e., \emph{de Sitter space}.

Why dS?
\begin{enumerate}
  \item Observation tells us that our spacetime is asymptotically dS at $t \to  \pm \infty $.
    \begin{equation*}
      \begin{aligned}
        t &\to -\infty ,\quad \text{inflation}, \\
        t &\to +\infty ,\quad  \text{dark energy}
      \end{aligned}
    \end{equation*}
  \item Particle physics: CMB is at $T = 2.7 \mathrm{K}$, while $\Delta T \simeq 10^{-5} \mathrm{K}$ is from the quantum fluctuation of space and matter fields during inflation.
  \item QFT in dS
  \item Amplitude
\end{enumerate}

\begin{figure}[ht]
    \centering
    \incfig{figure-of-ads}
    \caption{figure of spacetime}
    \label{fig:figure-of-ads}
\end{figure}
